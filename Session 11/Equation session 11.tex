\documentclass{article}
\usepackage[utf8]{inputenc}

\title{Equation classe}
\author{Maxime Walder}
\date{March 2022}

\begin{document}

\maketitle

\section{Introduction}

\bigskip

Moyenne (population):\par
\begin{equation}
    \mu = \frac{\sum_{i=1}^{N} X_i}{N}
\end{equation}

Moyenne (échatillon): \par


\begin{equation}
    \bar{X} = \frac{\sum_{i=1}^{n} X_i}{n}
\end{equation}

\bigskip

Étendue

\begin{equation}
    Etendue = X_{max} - X_{min}
\end{equation}

\bigskip


Éspace interquartile
\begin{equation}
    Interquartile = Q_3 - Q_1
\end{equation}


\bigskip


Écart type (population)
\begin{equation}
    \sigma = \sqrt{\frac{\sum_{i=1}^{N} (X_i - \mu)^2}{N}}
\end{equation}


Écart type (échantillon)
\begin{equation}
    S = \sqrt{\frac{\sum_{i=1}^{n} (X_i - \bar{X})^2}{n-1}}
\end{equation}


Variance (population)
\begin{equation}
    \sigma^2 = \frac{\sum_{i=1}^{N} (X_i - \mu)^2}{N}
\end{equation}


Variance (échantillon)
\begin{equation}
    S^2 = \frac{\sum_{i=1}^{n} (X_i - \bar{X})^2}{n-1}
\end{equation}

Skewness (Pearson-Fisher)

\begin{equation}
    Sk_1 = \frac{\frac{\sum_{i=1}^{n}(X_i - \bar{X})^3}{n}}{S^3}
\end{equation}

Skewness (Galton skewness)

\begin{equation}
    Sk_{Galton} = \frac{Q_1 + Q_3 -2Q_2}{Q3-Q1}
\end{equation}

Skewness (Person 2 Skewness)

\begin{equation}
    Sk_2 = 3\frac{\bar{x} + Q_2}{S}
\end{equation}

Kurtosis

\begin{equation}
    Kurt = \frac{\frac{\sum_{i=1}^{n}(X_i - \bar{X})^4}{n}}{S^4}
\end{equation}


Intervalle de confiance ($\alpha$ = 95\%)

\begin{equation}
    [\bar{X} - 1.96 * \frac{S}{\sqrt{n}}, \bar{X} + 1.96 * \frac{S}{\sqrt{n}}]
\end{equation}

\end{document}
