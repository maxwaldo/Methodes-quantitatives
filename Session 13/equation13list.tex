\documentclass{article}
\usepackage[utf8]{inputenc}

\title{Equation session 13}
\author{Maxime Walder}
\date{March 2022}

\begin{document}

\maketitle

\section{Equation list}

\bigskip


Correlation Pearson (version simplifiée)

\begin{equation}
    r = \frac{Cov(X,Y)}{S_x S_y}
\end{equation}

\bigskip

Correlation Pearson:

\begin{equation}
    r_p = \frac{\sum_{i=1}^{n} (X_i - \bar{X}) (Y_i - \bar{Y})}{\sqrt{\sum_{i=1}^n (X_i - \bar{X})^2}\sqrt{\sum_{i=1}^n (Y_i - \bar{Y})^2}}
\end{equation}

On a donc la covariance entre x et y multiplié par l'inverse de la racine carrée du produit des variances. 

\begin{equation}
    Corr = Cov(X,Y) \times \frac{1}{\sqrt{Var(X)Var(Y)}}
\end{equation}

Si on développe le calcule, on voit que les n peuvent se simplifier. 

\begin{equation}
    Corr(x,y) = \frac{\sum_{i=1}^{n} (X_i - \bar{X}) (Y_i - \bar{Y})}{n} \times \frac{\sqrt{n}\sqrt{n}}{\sqrt{\sum_{i=1}^{n} (X_i - \bar{X})^2}\sqrt{\sum_{i=1}^{n} (Y_i - \bar{Y})^2}}
\end{equation}

On a donc bien la correlation de Pearson:

\begin{equation}
    r_p = \frac{\sum_{i=1}^{n} (X_i - \bar{X}) (Y_i - \bar{Y})}{\sqrt{\sum_{i=1}^n (X_i - \bar{X})^2}\sqrt{\sum_{i=1}^n (Y_i - \bar{Y})^2}}
\end{equation}

\bigskip

Partial correlation 

\begin{equation}
    r_{xy.z} = \frac{r_{xy} - r_{xz}r_{yz}}{\sqrt{(1 - r_{xy}^2)(1 - r_{yz}^2)}}
\end{equation}







Régression linéaire univariée

\begin{equation}
    Y_i = \beta_0 + \beta_{1}x_i + \epsilon_
\end{equation}


\end{document}
